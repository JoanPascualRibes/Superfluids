\documentclass{article}
\usepackage{amscd,amssymb,amsmath,latexsym,graphicx,epsfig,color,url}
\usepackage{todonotes}

\usepackage{graphicx} % Required for inserting images

\newcommand{\red}{\textcolor{red}}
\newcommand{\blue}{\textcolor{blue}}
\newcommand{\black}{\textcolor{black}}
\newcommand{\green}{\textcolor{green}}

\newcommand{\Res}{\operatorname{Res}}
\title{Superfluids}
\author{
    Eloi Estèvez \and Eki González \and Joan Pascual \and Timothy Skipper
}
\date{\today}

\begin{document}

\maketitle
\begin{abstract}
    \todo{Write abstract}
\end{abstract}

\section{Introduction}

Qué es un superfluido?

Propiedades importantes/básicas

\section{Teoria básica de fluidos}

Navier-Stokes...

\section{Modelo de Landau}

% Faig alguns apunts random
The first macroscopic models of superfluid helium-4 were proposed by Tisza. His
final model consists of four evolution equations, namely, the continuity
equation, balance of momentum, evolution of superfluid velocity, and entropy
balance. However, the model has several limitations.  First, it does not allow
for non-zero superfluid vorticity (quantum vortices).

The Landau-Tisza model is formulated in terms of five quantities (superfluid
density $\rho_s$, normal density $\rho_n$, superfluid velocity $v_s$, normal
velocity $v_n$ and entropy density $s$). Since we have more variables than
equations we set a dependence of the ratio $\rho_n/\rho$ on temperature. But
this goes against the nature of superfluid helium-4, which is a single fluid
with two motions, as expressed by Landau: ``It must be particularly stressed
that we have here no real division of the particles of the liquid into
`superfluid' and `normal' ones\ldots.''

\section{Nuestra simulación}
Que métodos de resolución de EDPS usamos, que programas, resultados,
predicciones.

\section{Conclusión}

\section{Acknowledgements}

\begin{thebibliography}{XXX93}

    \bibitem[Eis95]{eis95} Eisberg, R., and Resnick, R.
    \newblock{\em Quantum Physics of Atoms, Molecules, Solids, Nuclei, and
        Particles./} Wiley, New York.

    \bibitem[KSP23]{ksp23} Kincl, O., Schmoranzer, D., and Pavelka, M.
    \newblock{\em Simulation of superfluid fountain effect using smoothed
        particle
        hydrodynamics./}
    \newblock{Physics of Fluids, 1 April 2023, 35(4): 047124.
        https://doi.org/10.1063/5.0145864}

\end{thebibliography}
\end{document}
