\documentclass{article}
\usepackage{amscd,amssymb,amsmath,latexsym,graphicx,epsfig,color,url}

\usepackage{graphicx} % Required for inserting images

\newcommand{\red}{\textcolor{red}}
\newcommand{\blue}{\textcolor{blue}}
\newcommand{\black}{\textcolor{black}}
\newcommand{\green}{\textcolor{green}}

\newcommand{\Res}{\operatorname{Res}}
\title{Superfluids}
\author{Eloi Estèvez, Eki González, Joan Pascual, Timothy David Skipper}
\date{\today}

\begin{document}

\maketitle
\begin{abstract}
[Write abstract]
\end{abstract}



\section{Introduction}

\subsection{Helium-II}
One of the most important properties of helium is that it cannot freeze at ambient pressure (25 atmospheres are required). It remains liquid for near-zero temperatures. The explanation for this unique behavior lies in the Heisenberg uncertainty principle, which gains importance for atoms with low mass and low potential, which perfectly matches the lightest of the noble gases.
\\

Pioneers in cryogenics were interested in the problem of minimizing the temperature of helium. The first to liquefy helium at 4.2 K was Heike Kamerlingh in 1908. A couple of years later, he realized that below 2.17 K the violent boiling process disappeared radically, although there is still phase change to vapor. The disappearance of the bubbles implies that there is no longer an irregular temperature distribution in the liquid. Now, if we place an electrical resistor in the helium below 2.17 K, it will dissipate the heat efficiently enough so that no bubbles appear. 
\\

This new state of helium became known as “Helium-II”. Willem Hendrik discovered that Helium-II was the best thermal conductor of all known materials, capable of flattening any thermal gradient. 
\\

On the other hand, Kamerlingh Onnes and Leo Dana found that cooling was more difficult near the transition because of the sudden increase in specific heat. The silhouette of the specific heat peak gave the phase change between Helium-II and Helium-I its name: the lambda transition, which is the sharpest phase transition known to us. This result can be related to conclusions from the study of the specific heat of Bose gases near Bose-Einstein condensate temperatures.
\\

\subsection{Superflow}

It was not until 1938 that the most important property of He-II was revealed.
Researchers Jack Allen and Donald Misener, on the one hand, and Pjotr Kapitza, on the other, conducted experiments on the flow of He-II to conclude that it can flow without viscosity.It turns out that at temperatures below the lambda temperature, this substance presents (with current experiments) no difficulty in passing through capillaries of the order of nanometers. This phenomenon was named superfluidity. The ideal fluid behavior adds to the arguments in favor of He-II having a Bose condensate of He atoms.
\\

We could conclude that this phenomenon is due to the disappearance of viscosity.However, the nature is not so simple.Experimentally, using a viscometer, a gradual (not sudden) drop in viscosity is observed starting at $T_\lambda$.
\\

\subsection{Andronikashvili torsional oscillator experiment}

This may appear to be a contradiction, but how can it exhibit two different behaviors? This mystery led Lev Landau to hypothesize the existence of a viscous and a superfluid component. However, these components do not imply distinguishable fluids, since the superfluid behavior is not eliminated when He-II is filtered through the capillary. Therefore, these components can transform into each other, as if each particle had both natures. This “duality” seems to point to quantum phenomenology: the Helium-II phase is a “quantum state” of matter.
\\

This duality was studied by Elepter Andronikashvili in 1946. He constructed a stack of finely separated disks, which he attached to the ceiling of the experimental cell forming a torsional oscillator. The frequency of oscillation is $\omega = \sqrt{\kappa/I}$, where $\kappa$ is the stiffness coefficient of the string and I is the moment of inertia of string and disks. By measuring the frequency we can find the moment of inertia. While the viscous fluid contributes to the moment of inertia, the superfluid component does not. Andronikashvili was thus able to measure the fraction that remained viscous.
\\

It is then concluded that the total density can be understood as the sum of superfluid component (non-viscous and does not allow temperature gradients) and a normal (viscous) component.
\\

\[\rho = \rho_s + \rho_n\].

The superfluid part appears at lambda temperature and increases its presence with decreasing temperature until at near zero temperature the normal part is negligible. The normal component has non-zero thermal resistance, but this acts in parallel with the other, resulting in the discontinuity of thermal gradients observed in $T_\lambda$.
\\

It should be emphasized that the fraction of each component is a function of temperature only. If you had more normal component concentration in one part of the fluid then this would imply a thermal gradient which we have concluded is impossible.
\\

\section{Teoria básica de fluidos}

Navier-Stokes...

\section{Modelo de Landau}
\cite{Kincl}
Main results of Landau:
Theory of liquid helium. Introduction of notion of quasiparticles. Superfluid
helium as a quantum liquid. Phonons and rotons. Prediction of second, third,
fourth and fifth sounds, zero sound.

\cite{PhysRev.60.356}

% Faig alguns apunts random
The first macroscopic models of superfluid helium-4 were
proposed by Tisza. His final model consists of four evolution equations, namely, the continuity equation, balance of momentum, evolution of superfluid velocity, and entropy balance. However, the model has several limitations. First, it does not allow for non-zero superfluid vorticity (quantum vortices). 

The Landau-Tisza model is formulated in terms of fice quantities (superfluid density $\rho_s$, normal density $\rho_n$, superfluid velocity $v_s$, normal velocity $v_n$ and entropy density $s$. Since we have more variables than equations we set a depencence onf the ratio $\rho_n/\rho$ on temperature. But this goes against the nature of superfluid helium-4, which is a single fluid with two motions, as expressed by Landau: “It must be particularly stressed that we
have here no real division of the particles of the liquid into ‘superfluid’
and ‘normal’ ones.”


\section{Nuestra simulación}
Que métodos de resolución de EDPS usamos, que programas, resultados, predicciones.

\section{Conclusión}


 
\section{Acknowledgements}


\

\bibliographystyle{alpha} % Choose a style (e.g., plain, alpha, IEEEtran)
\bibliography{bibliography} % Point to your .bib file (without extension)


\end{document}
